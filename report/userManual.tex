\documentclass[italian,12pt,a4paper,oneside,final]{report}
\usepackage[toc]{appendix}
\usepackage{listings}
\usepackage{graphicx}
\usepackage[utf8]{inputenc}
\usepackage[italian]{babel}
\usepackage{csquotes}
\usepackage[skins]{tcolorbox}
\usepackage{hyperref}
\graphicspath{ {images/} }
\lstset{captionpos=b,showspaces=false,basicstyle=\ttfamily,showstringspaces=false,breaklines=true}
\renewcommand{\thesection}{\arabic{section}} % remove the \chapter counter from being printed with every \section
\renewcommand{\appendixtocname}{Appendice}
\renewcommand{\appendixpagename}{Appendice}
\hypersetup{
	colorlinks=true,
	linkcolor=,
	urlcolor=blue,
	pdftitle={Marco Giunta - Progetto CS},
	pdfauthor={Marco Giunta},
}
\newtcolorbox{mytcolorbox}[1]{
	enhanced,
	colback=white,
	colframe=gray,
	coltitle=black,
	fonttitle=\bfseries,
	detach title,
	boxrule=2pt,
	leftrule=0pt,
	top=1mm,
	attach title to upper={\par},
	sharp corners,
	borderline west={5pt}{0pt},
	title=#1
}

\title{\Large Piattaforma per la distribuzione di\\biglietti gratuiti per eventi\\[0.5em]
	\large Istruzioni di installazione}
\date{Giugno 2025}
\author{
	Marco Giunta\thanks{Marco Giunta 147852 giunta.marco@spes.uniud.it}\\
	Cybersecurity 2024/25, Università degli Studi di Udine}
	
\begin{document}
% Generate title page
\maketitle
	
% Generate TOCs
\pagenumbering{arabic}
\tableofcontents
\section{Introduzione}
In questo documento ci sono le istruzioni per l'installazione dell'applicazione web ``Piattaforma per la distribuzione di biglietti gratuiti per eventi''.

\section{Prerequisiti}
Prima di procedere all'avvio dell'applicazione sono necessari i seguenti software:

\begin{itemize}
	\item Docker  (\url{https://docs.docker.com/get-started/get-docker/})
	\item Docker Compose (\url{https://docs.docker.com/compose/install/})
\end{itemize}

\section{Installazione}
Copiare il file \emph{./docker/.env.dist} in \emph{./docker/.env} e modificare le variabili: \newline

\begin{lstlisting}
PROJECT_ROOT=/PATH/TO/PROJECT
PHP_PEPPER=RANDOM_32BYTES_HEX_STRING
PHP_SECRET=RANDOM_32BYTES_HEX_STRING
\end{lstlisting}
\hfill \break
\noindent \textsc{/PATH/TO/PROJECT} deve essere il percorso \emph{assoluto} della directory del progetto. \newpage
\noindent Per generare una stringa \textit{random} di 32 Bytes si può usare il comando:
\begin{lstlisting}[language=bash] 
openssl rand -hex 32
\end{lstlisting}

\begin{mytcolorbox}{Problema di permessi su MacOS}
Per evitare problemi di permessi causati da UID e GID diversi tra l'host e i \textit{container}, utilizzare la versione \textbf{-dev-macos} dell'immagine PHP (rimuovere il commento dalla variabile \textbf{PHP\_TAG} nel file .env) dove l'utente predefinito ha UID/GID 501:20 che corrisponde all'utente predefinito di MacOS.
\end{mytcolorbox}
\hfill \break
\noindent Avviare i \textit{container} con il comando:
\begin{lstlisting}[language=bash]
cd docker
docker compose up -d
\end{lstlisting}

\section{Configurazione}
Durante il primo avvio del container \emph{mariadb}, il database viene popolato con i dati presenti nella directory \textit{./mariadb-init}. \newline
In particolare, vengono creati due account:

\begin{itemize}
	\item admin@example.com, con privilegi di amministratore
	\item user@example.com, con privilegi utente
\end{itemize}

\noindent Per impostare una password a  \textbf{TUTTI} gli utenti presenti nel db, usare il comando:

\begin{lstlisting}
docker exec -ti tickets_cs_project_php php scripts/change_default_passwords.php PASSWORD
\end{lstlisting}

\noindent Dopo aver impostato la password, collegarsi al sito \url{https://localhost} .

\end{document}